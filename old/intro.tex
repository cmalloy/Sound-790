[intro]

Need to cover:
\begin{itemize}
\item The goal of the project
\item Prior work
\item Results 
\item List of open issues 
\end{itemize}



For our project we intend to leverage feedback from the auralization (and rendering) stage of the 3d audio system for more efficient sound propagation. We would like to use psychoacoustic metrics (with an initial emphasis on temporal/intensity masking) to prioritize the rendering of paths and to better inform the sound propagator on how to best distribute rays to sources and sample the scene.

Our's is related to the work of Tsingos et al (i.e. in Perceptual Audio Rendering of Complex Virtual Environments) although we intend to leverage psychoacoustic feedback throughout the entire pipeline (including propagation). There has also been a lot of related work in audio compression, for example the MPEG series of codecs and in the compression techniques used in video conferencing / audio streaming.

Our progress so far has been:

 reviewing literature on sound propagation, psychoacoustics, audio compression, and Tsingos work 
code reviews of gSound
putting together a simple test scene for experimenting w/ gSound in Unity
preliminary design of the system as far as integrating it with gSound
We intend to continue by reproducing the methodology of Tsingos and using perceptual feedback from rendering/auralization to better distribute rays to sources. We also have a design for coupling this with gSound. After experimenting with the validity of this approach in accelerating propagation, our current idea is that we will then try to further leverage this information as much as possible during propagation (i.e. sampling the hemisphere around sources/listeners based on perceptual feedback, importance sampling similar to what is done in path based light renderers, and/or transitioning to a more progressive renderer).

Also, our project site is: Our project site is: https://sites.google.com/a/cs.unc.edu/sound-sim-project/



Goal

Investigate perceptually-based optimizations of path-based sound propagation and rendering, particularly by using information available during the auralization and rendering phase for importance.


Motivation

To prioritize perceptually important paths and leverage information about the relative contribution of real and virtual sources throughout the propagation and rendering pipeline.

Schedule: What I had in mind was the following:

(1) Become familiar with gSound enough to modify sound propagation and rendering, centered around the SoundPropagator and SoundPropagationRenderer classes and their interactions in the pipeline. Also design a modified pipeline for bidirectional tracing w/ importance sampling.
(2) Add HRTF support to SoundPropogationRenderer
(3) Evaluate the existing efficiency of the pipeline based on the distribution of paths in the existing renderer
(4) Decouple the propagation to include both backward and forward paths (carrying importance and source information respectively)
(5) Try strategies to connect the paths and guide sampling on both ends to converge more efficiently
(6) Demo