This project aims to produce auditory display pipeline with perceptually optimal sound propagation component (subsystem, state). We force a simulator to focus its computational resources on the perceptually salient regions of an audio (aural/sound) scene, while maintaining the high (same) level of perceptual quality and remaining computationally efficient.

The sound simulation is important area of research and application. Interactive sound simulation is used in many field (games, VR, architectural acoustics). Computation of the sound propagation in complex scenes is either expensive or imprecise. Wave based sound propagation methods can produce accurate results \cite{waves}, but take too much time to be considered for real-time applications. Geometry based methods, on the other hand, provide fast, but approximate solution to the wave equation. The accuracy of geometric (ray-based) approaches can be improved by shooting (generating) more rays (paths), however this directly correlates with the computation time of the simulation. The core idea of our project is to exploit the perceptual limitations of the human auditory system to shoot the rays robustly.

[Small intro on perceptual part ...] It is reasonable to use this psychoacoustic principles to guide the propagation system during ray sampling phase.

We introduce a combination of techniques that improve the running time or/and perceptual quality of the sound propagation system. We first compute perceptual loudness corresponding to the sound signal and the impulse response (IR) of each path from the listener to the sound source. We next use the directional information of the paths to generate spherical distribution of a loudness around the listener, which, combined with HRTF, characterizes how an ear receives a sound from a point in space. We finally integrate the spherical loudness distribution into the next frame's sampling strategy.

Concretely, following are the contributions of our project work:
\begin{itemize}
\item a novel approach of using the psychoacoustic principles during the sound propagation stage;
\item a new technique for constructing loudness distribution map;
\item a partial preprocessing of the perceptual information for maintain real-time level of computations. 
\item an implementation the proposed methods and integration of them into gSound, an interactive sound propagation and rendering system.
\end{itemize}

Operating together these contributions constitute a system that can produce a Jedi from any hopeless Padawan.

% \subsection{Goal}
% Investigate perceptually-based optimizations of path-based sound propagation and rendering, particularly by using information available during the auralization and rendering phase.

% Exploit perceptual limitations of the human auditory system. Compression due to removal of perceptually irrelevant part of the signal. Results in inaudiable distortion.

% The human auditory system's inability to to hear quantization noise under condition of auditory masking.

