The MPEG-1 audio standard specifies an algorithm for compressing digital audio signals by removing the perceptually inaudible components. The standard defines two psychoacoustic models that model the human sound perception system and inform the quantization of individual audio blocks during compression. The models differ in computational complexity, but share the main idea of splitting sound into tone-masking-noise and noise-masking-tone, and calculating a global masking threshold curve  by combining the two individual models based on how tonal or noise-like the signal is. This global masking threshold represents the minimal sound power perceptible by the average human listener. The global threshold guides the encoders allocation of bits to better represent an error-free signal. In the psychoacoustic model 1 the tonality of a component is determined from peaks in the critical bands, while in the model 2, it is determined using a predictability measure.
 
 The psychoacoustic models involve following main steps of the global masking threshold computation:
 \begin{itemize}
 \item Spectral Analysis: calculation of the \emph{power spectral density} (PSD) which describes the distribution of the power of a signal over the different frequencies.
 \item Critical Band Analysis: the power spectrum of the signal is partitioned into critical bands by integrating over the corresponding bandwidths.
 \item Tonality Analysis: the \emph{spectral flatness measure} (SFM) is used to characterize the spectrum of the signal. The SFM is defined as the ratio of the geometric to arithmetic mean values of power spectral density. The \emph{tonality index} (TI) is calculated by comparing estimated SFM with the SFM of a sinusoidal signal. Values of the tonality index that are closer to zero indicate that the spectrum of the signal look similar to a spectrum of white noise. Values of TI that are closer to one indicate that the spectral power is concentrated in a relatively small number of bands and the signal sounds like a mixture of sine waves.
 \item Masking Analysis: The masking threshold is determined by calculating an offset to the excitation pattern. Due to asymmetry of tone and noise maskings the value of offset depends on the value of the tonality index. The final values for the offset are interpolated from the offsets for tone-like and noise-like signals.
 \item Spread of Masking: Inter-band masking is taken into account by convolving each of the maskers with the \emph{spreading function}.
 \end{itemize}
